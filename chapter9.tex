%===================================== CHAP 9 =================================

\chapter{Discussion and Conclusion}
The outcome of the Design and Creation research strategy used for this thesis is a combination of the working system, methods, models and constructs. Once combined, these IT artefacts can offer knowledge to the research field \cite{oates2005researching}. In this section we will discuss the research question presented in chapter \ref{chapter:1} in relation to the artefacts and the analysis results from the final phase. Limitations of the study and contributions is also discussed.
Finally, we present our conclusions and recommendations for future work.


\section{Discussion}

\subsection{The artefact} 
Table \ref{table:comparisonOurApp} lists and compares features from the related work applications presented in chapter \ref{chap:relatedWork} and the features which was implemented in our artefact supporting a collaborative VR environment aimed at facilitating career guidance. As evident by the table, the artefact of which was created as part of this thesis supports all the features we identified in the preliminary research study to be important in hopes of answering the research questions set. The following sections will discuss their value and effect.   
\\
\\ "\ON" = has the feature.
\\ "\LIM" = has the feature, but is limited.

\begin{table}[]
    \begin{center}
    \begin{tabular}{@{}l c c c c @{}}
           & \multicolumn{3}{c}{\textbf{Related Work}}
    \\  \cmidrule{2-4}
           & \textbf{Workplace}
    \\       
             \textbf{Features}
           & \textbf{Internship}
           & \textbf{ElectroVR}
           & \textbf{CoVAR}
           & \textbf{Our application}
    \\ \midrule
       VR                           & \ON & \ON  & \ON  & \ON
    \\ Multiplayer                  &     & \LIM & \LIM & \ON
    \\ Workplace training           & \ON &      &      & \ON
    \\ Collaboration                &     & \ON  &      & \ON
    \\ Real-world simulation        & \ON & \LIM & \ON  & \ON
    \\ Voice-chat                   &     &      &      & \ON
    \\ Co-located                   & \ON & \ON  & \LIM & \ON
    \\ Remote                       &     &      & \ON  & \ON    
    \\ Symmetric role               & \ON & \ON  & \LIM & \ON  
    \\ Asymmetric role              &     & \ON  & \LIM & \ON
    \\ \bottomrule
    \end{tabular}
    \captionsetup{width=1\linewidth}
    \caption{Related work applications and their features compared to our implementation.}
    \label{table:comparisonOurApp}
    \end{center}
\end{table}


\subsection{Research Questions}  
\label{RQDiscussion}

\subsubsection{Primary RQ: \textit{How does collaboration in virtual reality workplaces contribute to the career guidance of young job seekers?}} 

The primary goal of this research paper has been to discover whether or not collaborative VR is conducive to career guidance. Both the primary qualitative data and the secondary quantitative data that was gathered has a high degree of relevance concerning this question. 

First and foremost, how did actual NAV clients and NAV employees compare the collaborative VR experience to a normal VR experience? While the number of respondents and testers were not as high as wanted, even for a qualitative study, the opinions of the actual users of the system were extremely important. Not only had they tested the single-user VR experience already, many of them were currently in or had been in  a situation where they needed NAV help to find employment. Looking at the qualitative data, we can see that the responses have been quite positive. Table \ref{table:phase3SatisfactionAnalysis} shows the perceived value of the concept based on interviews with the testers. One of the points brought up most often here is the concept of \textit{engagement}. Looking at table \ref{table:phase3MethaphorsAnalysis} which shows metaphors and keywords that appeared, engagement is one of only three which was used in every interview. 

The engagement of users is an important point that 

Foundation for VR: Immersion (HMD section => resolution, refresh rate, FOV) and presence, social presence, Workspace Awarenes, CSCL

VR Collab, career guidance, experiential learning, gamification (balancing act, should consider care of having to much gamification for young job seeker at NAV since it might impact their self-efficacy in a negative way) 

Data results: 


\subsubsection{Secondary RQ1: \textit{How is self-efficacy affected by seeker-seeker collaboration compared to seeker-councillor collaboration in virtual reality?}} 

Several features were implemented and numerous aspects were considered so that the artefact could support a wide range of different use cases. This includes a traditional career guidance situation with one seeker and a councillor, but is not limited to it. The application system can handle any number of seekers and councillors (up to 16) at any given point, using either VR equipment or a traditional desktop. This means users can utilise the application in a great number of different circumstances. For instance, four young job seekers situated in Trondheim can join a virtual workplace environment from their NAV facility or at home, while a business manager situated in Oslo can show how they operate whilst at the same time a councillor observers the situation.           
Since the application supports both symmetric and asymmetric roles, plus remote or co-located usage, which based on our research of other applications does not exist for career guidance cases, it would be very interesting to see how this could affect young job seekers and their self-efficacy. 


As previously described, the Covid-19 pandemic heavily impacted our ability to collect data material in regards to this research question. Described in section \ref{section:covid19} the restrictions by the Norwegian government meant proper data collection was not feasible. We were not able to collect the desired amount of semi-structured interviews from both seeker-seeker and seeker-councillor pairs. The sample size meant the data material was not suitable for making generalisations of the target group, which generally requires a larger sample size \cite{oates2005researching}. We could there not justify any valid discussion or conclusions related to the research question since we did not possess reliable and rich data.                

This research question will thus become a suggestion for future work. See section \ref{section:futureWork}.  

%Explain why we cannot discuss and conclude this RQ. Lacking data material due to Covid-19. 

%Career guiders from NAV Sandefjord in the interview mentioned that youths often chooses what friends do when selecting a career path. They do however think career guides or even better having business leaders inside the VR app with youths could be good.
%\textcolor{red}{THIS IS NOT IN THE ANALYSIS FOR CHAPTER 7 SO IT MUST BE PUT IN THERE IF WE DISCUSS IT}

%This RQ becomes a suggestion for future work


\subsubsection{Secondary RQ2: \textit{Which features are effective at facilitating collaborative virtual reality for remote
career guidance?}} 
There are many concepts which are important to consider and contemplate when regarding the use of collaborative VR for remote applications. First and foremost, the subjective feeling of presence, especially social presence, should be considered the bedrock for VR if it is to be applied effectively for collaboration. A low feeling of social presence greatly impacts the overall experience of the application amongst its users. Factors such as visual representations of users (avatars), interaction with other agents and more behaviourally realistic visual representation has been shown to increase both involvement and engagement \cite{skalski2007role} \cite{oh2018systematic}. 

Interestingly, as shown in our analysis, figure \ref{fig:phase3_SurveyPrecenceIncrease} illustrates that the experts evaluation regarding the affect of social presence on the experience is divided and does not directly align with the research theory. This might be a direct result of experts lack of knowledge in the VR field, as the concept of presence is often misunderstood as being immersed. However, if we look at figure \ref{fig:phase3_SurveyValgkompVideo} we can see that a very similar statement: \textit{To what degree do you feel the presence of another user is useful for developing career decision learning?} was ranked \textit{High degree} by 63.6\%. Although not similar, these statements are concerned about the same concept (social presence), but they show different results depending on how they experienced the application. It is evident they perceive the value of presence differently, and should be taken into consideration. The being said, these data results are regarded as support material due to the circumstances explained before and in an ideal world all testing would have been conducted using VR headsets or desktop mode so that they get a proper feel for the artefact. 
Noticeably, as mentioned in the analysis, figure \ref{fig:phase3_socialPresence} shows experts perceived a relative high degree of presence and social presence from the artefact. Although a few rankings were sub neutral there is a clear common consensus that they feel access to another intelligence and its intentions.

Workspace awareness is important for how users can collaborate efficiently \cite{gutwin1996workspace}. Thus elements presented in table \ref{table:awarenessPresent}, \ref{table:awarenessPast}  and \ref{table:awarenessActivity} are all especially vital to consider for any application to be used in a remote setting. If a user lacks awareness of another user's intentions then they have trouble understanding what goal and outcome actions they execute is part of. Resulting in a disorienting experience for both users. The awareness of present actions is particularly crucial for councillors observing other seekers in the workspace. Also, the ability to to explain how tasks work in order to provide assistance within the same virtual environment is a clear benefit compared to existing NAV applications where instructions was coming from the outside environment, resulting in a disconnected and difficult experience. With the artefact created for this thesis councillors can effortlessly communicate with all users (e.g job seekers) using the built in voice communication component while simultaneously being present in the same virtual environment, which according to table \ref{table:awarenessActivity} benefits the workspace awareness. Experts heavily agreed that verbal communication is a useful quality if the artefact is to be used for remote guidance as seen in figure \ref{fig:phase3_VerbalWritten}.         



%It is important to understand that remote career guidance could involve many different use cases depending on the situation. table 7.18

How developers utilises the advantages of VR embedded in the technology is decisive, as it can cater to and empower users that are not situated in the same geographical area to collaborate, share and learn about a workplace or career direction. During the development of artefact we included the ability to apply deictic gestures to demonstrate and collaborate. This is important as it increases the learning effect \cite{stahl2006computer} and should work seamlessly despite any geographical or networking hindrances. If not, this could heavily distort the artefacts ability to facilitate remote usage.               


For any artefact to be effective for remote usage, we assumed the the process of installing, setup and application use must be easy so that as little as possible of technical support is needed. No one wants to use a system which is hard or difficult to use. It simply becomes too much of a burden so that other solutions such as video conferences are used. As such, during the development we put considerable effort into making sure the artefact could be as intuitive and easy to use as the time-frame allowed. Table \ref{table:phase3MethaphorsAnalysis} and figure \ref{fig:phase3_Intuitiveness} show that both survey participants and interviews generally agree that the artefact is intuitive and has a high degree of ease of use. However, there are inconsistencies with the survey rankings and the SUS score, which provides a scale on the usability of the artefact. The average  SUS score yielded a poor rating of the usability, meaning there is potential for improvement. An important note is that all participants of the survey (which gave the SUS score) was all in the age group of 45-64 and 65\% of those had little to none experience with VR. This could have impacted the results but there is no way of conclusively knowing without doing further studies.  

With VR technology constantly evolving, allowing HMDs such as the Oculus Quest to disregard the use of a separate computer to hold the VR engine, the ease of use could be further improved. By publishing the application to the Oculus store it could be easier for the users to run the application, but also increase the availability, which was a keyword found in our theme analysis from the interviews (see table \ref{table:phase3MethaphorsAnalysis}). More on this in section \ref{section:futureWork}.
Although VR becomes more available to users, there is still a large percentage who does not have the means or access to the equipment needed to experience a VR environment. With the inclusion of the desktop mode in the artefact and through gamification principles we hoped that this could lower the threshold for the artefact to be used in remote settings. From the survey (where half used desktop mode) it was quite evident that this was supported by the experts as there was a high agreement that such an application could be useful for remote occupational guidance (see figure \ref{fig:phase3_AppBeUsed}). This reasoning is also supported by career guides in the interview as seen in table \ref{table:phase3SatisfactionAnalysis}. However, the desktop mode does not have the functionality VR mode has, meaning interaction with the environment is very limited, and might impact the experience. The intent of the desktop mode was primarily to let councillors observe seekers in the workplace so that not much functionality was put into the feature. Seeing its potential this could be improved further and perhaps become part of an future IMTEL project. 

To summarise, when facilitating collaborative VR for remote
career guidance it is important to factor in both fields of study such as CSCL, presence and interaction design while catering for the found effective features such as voice communication, gesticulation and visibility of users as avatars or other representations.   

%social presence
%CSCL
%workspace awareness
%Table 2.1-3

%for it to be effective in use it must be ease to install/setup/use by remote users with minimal support. technologies (VR hardware, Quest would be useful, can be future work)

%desktop mode and VR mode. Desktop mode can lower the threshold (gamification => uses known game principals such as WASD movement). Con: now interaction with VR workspace.



\subsubsection{Secondary RQ3: \textit{What type of collaborative features are technologically feasible for workplace experience in a virtual environment?}} 
\label{discussion:RQ3}




%Features (such as laserpointer or hand gesticulation) that can be developed and carried out to fulfil its objective.
%Technologies sections: Unity, PUN2, syncing movement, cross compatible with most HMDs.  

%fullbody avatar, uncanny valley 



\subsubsection{Secondary RQ4: \textit{What challenges arise when implementing collaborative features in an ongoing
single-user virtual reality project?}} 

As described in the chapter \ref{chapter:1}, this thesis could be regarded as a branch project of the Virtual Internship project. No previous projects have been develop with multi-user capabilities or with that in mind. There was therefore a wide range of obstacles and challenges which was encountered during the three phases of software development. First of all, it is the fact that single-user VR projects does not need to be concerned with anything related to networking. This means the foundations of an single-user project are often not suited for it be evolved into a multi-user application which supports collaborative features. What we found during the development was that there were little to no project structure or followed code standards. It was obvious the application selected was a students bachelors project were Unity and VR development was new for most of them. Thus the lacking structure and inconsistencies should be considered an expected consequence. In case of IMTEL's future projects there could be created a basic template for how development should be done. This is however difficult to adhere to as projects differ but should at least provide some means of similarity between projects resulting in a better foundation if they were to further developed as multi-player applications.    

There a few technical prerequisites that need to be in place before development can begin to implement collaborative features. However, some aspects determine the intricacies of implementation. These include the complexity of the application and underlying game elements, cohesion of code, number elements which needs networking and game state. With respect to technical requirements, there needs to be at least four things in place. First, (1) a  client-network architecture for data transmissions, (2) a lobby system for join or creating instances of scenes, (3) code (scripts) to handle networked objects and finally (4) managers for for keeping track of game state and general network related requests.
A part goal of this project was to create a general framework (including prefabs, scripts and lobby system) for future use. With this framework the difficulty of accommodate collaborative features drastically reduces. To further reduce difficulty we created an demo application in our artefact so that it could be used as a guide for development and implantation.    

Consideration must also be made towards what type of project benefits and is suitable for collaborative features. As an example is the Wind-turbine electrician application. Here there are limitations in regards to virtual space within to the wind-turbine so that having multiple users could present a crammed feeling for the users resulting in a negative experience. Also, the PUN2 cloud service used in the project has a limit of 16 concurrent users (CCU). If an application needs to support more than that there are basically two options: pay for more capacity in the Photon cloud or change framework. However, having more than 16 CCU is an unlikely use case for career guidance, the aim of the Virtual Internship is not to be a massively multiplayer online game (MMO).          

From experience there should also be at least two developers devoted to the task as it is very challenging to test collaboration features, or any other networking solutions with one developer. Additionally, for implementation purposes there should be an agreement of utilising a common VR SDK. SteamVR is a great choice as it enables support for building applications for major VR headset brands with little change. This yields a higher degree of user which can utilise the collaborating features, although some configuration could be needed to support similar actions by different button inputs due to differences in controllers amongst suppliers.  


%IMTEL standards/prefab for future use. A foundation for future projects
%Discussion about general difficulties
%Thoughts to consider: type arbeidsplass (lite plass i vindølle for flere), mengde brukere (opp til 16 pers, mer enn det må man betale for PUN cloud), synkronisering (komplekse oppgaver med states er krevende) => men finnes metoder som helper med dette (RPC)
% bør være minst to stykker for utvkling, da testing av multiplayer er vanskelig som en

%ProgArk pensum: high cohesion, low coupling osv., code standard
%https://medium.com/unit2games/making-multiplayer-games-doesnt-have-to-be-difficult-d08d19c83de3


\section{Contributions}
Through this project, we have created an artefact that can be used for future work at the IMTEL lab. Primarily, this includes a lobby system that can be made to work with any number of Unity scenes, prefabs that hopefully will make developing multi-user functionality easier and a template for how to best create a multi-user VR experience. 

\textit{Trenger noe om research contributions}

Artefact
- Lobby system
- Prefabs for VR/Desktop users
- can be used as a template for future work at IMTEL and NAV

Research-and-Development to the IMTEL and NAV partnership:
- findings in this thesis contributes to this


\section{Limitations}
While the overall project has been successful and produced a functional artefact as planned, testing was impacted by Covid-19. As a direct result of the pandemic, the amount of testers decreased, which also may have lead to some selection bias.

Originally, the plan was to test with NAV clients, which would allow for a good distribution of testers. This would include people from the entire spectrum of familiarity with technology and make sure that the artefact was usable to everyone, not just those who have used VR before. With the stay-at-home situation, testers were significantly harder to gather. Those that agreed to join us for remote testing already had VR headsets at home, or at least an interest in the technology, which introduces bias in the tester pool on top the already small selection. There was also no opportunity to test the artefact in schools.

Covid-19
Data gathering
Limited testers to NAV clients, not tested at schools
- Small scale tests
- Alle testere hadde VR headset, eller var involverte med prosjektet fra før av (Bias)





\section{Conclusion}
\textit{Conclusion}

\section{Future Work}
\label{section:futureWork}

\subsection{Secondary RQ1}
Secondary RQ1 as future work since we cannot definitely answer it now due to the Covid-19 situation.

%Gaze, as is in coVAR and SteamVR has gazescripts? i think  

\subsubsection{New hardware}
In order to make it easier to test quickly and easily, we recommend making use of hardware such as the Oculus Quest(\textcolor{red}{Legge til en ref til quest?}) in order to minimise the amount of hardware needed to run the applications. Our experience while testing showed that many schools and corporations had bought Oculus Quest in a large number, but had relatively few standard VR headsets, much less computers to run them on. 

Alternatively, more can be done to make use of the more expensive "standard" VR headsets, such as the Valve Index, with increased feedback and detailed hand gestures to increase realism and user presence.

\subsubsection{Desktop Mode}
While the desktop mode was not originally planned, it's inclusion garnered positive feedback and a desire to see more functionality added to it. Expanding it's features and allowing greater interaction with the scene while in this mode could increase the availability and usability of the artefact, and allowing a larger variation of use scenarios. The time allotted for this project did not allow for a thorough examination of potential features, but preliminary feedback would indicate that the most desired features are a laser pointer, the ability to pick up objects, and more highlighting opportunities.

Depending on the extension of this mode, it may also be relevant to change the avatar and movement mode to something more akin to the VR players, i.e. a humanoid avatar and movement affected by physics constricted to ground level. It would be interesting to see how this affected collaboration and whether or not it is jarring for a VR user to collaborate with a humanoid avatar with less natural movement and behaviour like a desktop avatar would bring to the table.


\cleardoublepage