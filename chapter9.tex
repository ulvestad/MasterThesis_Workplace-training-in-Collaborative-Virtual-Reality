%===================================== CHAP 9 =================================

\chapter{Discussion and Conclusion}
The outcome of the Design and Creation research strategy used for this thesis is a combination of the working system, methods, models and constructs. Once combined, these IT artefacts can offer knowledge to the research field \cite{oates2005researching}. In this section we will discuss the research question presented in chapter \ref{chapter:1} in relation to the artefacts and the analysis results from the final phase. Limitations of the study and contributions is also discussed.
Finally, we present our conclusions and recommendations for future work.


\section{Discussion}

\subsection{The artefact} 
Table \ref{table:comparisonOurApp} lists and compares features from the related work applications presented in chapter \ref{chap:relatedWork} and the features which was implemented in our artefact supporting a collaborative VR environment aimed at facilitating career guidance. As evident by the table, the artefact of which was created as part of this thesis supports all the features we identified in the preliminary research study to be important in hopes of answering the research questions set. The following sections will discuss their value and effect.   
\\
\\ "\ON" = has the feature.
\\ "\LIM" = has the feature, but is limited.

\begin{table}[]
    \begin{center}
    \begin{tabular}{@{}l c c c c @{}}
           & \multicolumn{3}{c}{\textbf{Related Work}}
    \\  \cmidrule{2-4}
           & \textbf{Workplace}
    \\       
             \textbf{Features}
           & \textbf{Internship}
           & \textbf{ElectroVR}
           & \textbf{CoVAR}
           & \textbf{Our application}
    \\ \midrule
       VR                           & \ON & \ON  & \ON  & \ON
    \\ Multiplayer                  &     & \LIM & \LIM & \ON
    \\ Workplace training           & \ON &      &      & \ON
    \\ Collaboration                &     & \ON  &      & \ON
    \\ Real-world simulation        & \ON & \LIM & \ON  & \ON
    \\ Voice-chat                   &     &      &      & \ON
    \\ Co-located                   & \ON & \ON  & \LIM & \ON
    \\ Remote                       &     &      & \ON  & \ON    
    \\ Symmetric role               & \ON & \ON  & \LIM & \ON  
    \\ Asymmetric role              &     & \ON  & \LIM & \ON
    \\ \bottomrule
    \end{tabular}
    \captionsetup{width=1\linewidth}
    \caption{Related work applications and their features compared to our implementation.}
    \label{table:comparisonOurApp}
    \end{center}
\end{table}


\subsection{Research Questions}  
\label{RQDiscussion}

\subsubsection{Primary RQ: \textit{How does collaboration in virtual reality workplaces contribute to the career guidance of young job seekers?}} 

Foundation for VR: Immersion (HMD section => resolution, refresh rate, FOV) and presence, social presence, Workspace Awarenes, CSCL

VR Collab, career guidance, experiential learning, gamification (balancing act, should consider care of having to much gamification for young job seeker at NAV since it might impact their self-efficacy in a negative way) 

Data results: 


\subsubsection{Secondary RQ1: \textit{How is self-efficacy affected by seeker-seeker collaboration compared to seeker-councillor collaboration in virtual reality?}} 

Explain why we cannot discuss and conclude this RQ. Lacking data material due to Covid-19. 

Career guiders from NAV Sandefjord in the interview mentioned that youths often chooses what friends do when selecting a career path. They do however think career guides or even better having business leaders inside the VR app with youths could be good.
\textcolor{red}{THIS IS NOT IN THE ANALYSIS FOR CHAPTER 7 SO IT MUST BE PUT IN THERE IF WE DISCUSS IT}

This RQ becomes a suggestion for future work


\subsubsection{Secondary RQ2: \textit{Which features are effective at facilitating collaborative virtual reality for remote
career guidance?}} 

social presence
CSCL
workspace awareness
Table 2.1-3

for it to be effective in use it must be ease to install/setup/use by remote users with minimal support. technologies (VR hardware, Quest would be useful, can be future work)

desktop mode and VR mode. Desktop mode can lower the threshold (gamification => uses known game principals such as WASD movement). Con: now interaction with VR workspace.



\subsubsection{Secondary RQ3: \textit{What type of collaborative features are technologically feasible for workplace experience in a virtual environment?}} 


Features (such as laserpointer or hand gesticulation) that can be developed and carried out to fulfil its objective.
Technologies sections: Unity, PUN2, syncing movement, cross compatible with most HMDs.  

fullbody avatar, uncanny valley 



\subsubsection{Secondary RQ4: \textit{What challenges arise when implementing collaborative features in an ongoing
single-user virtual reality project?}} 

Discussion about general difficulties
Thoughts to consider: type arbeidsplass (lite plass i vindølle for flere), mengde brukere (opp til 16 pers, mer enn det må man betale for PUN cloud), synkronisering (komplekse oppgaver med states er krevende) => men finnes metoder som helper med dette (RPC)
ProgArk pensum: high cohesion, low coupling osv., code standard


\section{Contributions}

Artefact
- Lobby system
- Prefabs for VR/Desktop users
- can be used as a template for future work at IMTEL and NAV

Research-and-Development to the IMTEL and NAV partnership:
- findings in this thesis contributes to this


\section{Limitations}
Covid-19
Data gathering
Limited testers to NAV clients, not tested at schools
- Small scale tests





\section{Conclusion}
\textit{Conclusion}

\section{Future Work}
\label{section:futureWork}

\subsection{Secondary RQ1}
Secondary RQ1 as future work since we cannot definitely answer it now due to the Covid-19 situation.

%Gaze, as is in coVAR and SteamVR has gazescripts? i think  

\subsubsection{New hardware}
Oculus Quest 

\subsubsection{Desktop Mode}
Full extension of this mode. Interaction, perhaps more realistic representation (avatar)

\cleardoublepage