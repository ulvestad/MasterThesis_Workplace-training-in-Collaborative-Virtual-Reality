%===================================== CHAP 1 =================================

\chapter{Introduction}
\label{chapter:1}
This project explores the use of virtual reality (VR) to create a collaborative environment for career guidance. In this chapter we present the context, motivation and research questions of the project. 

\section{Context}
\label{section:context}
This thesis is part of a master's programme in Informatics at the Norwegian University of Science and Technology (NTNU) in Trondheim. It is a collaboration between the Department of Computer Science (IDI), the Department of Education and Lifelong Learning (Innovative Immersive Technologies for Learning, IMTEL) and the Norwegian Labour and Welfare Administration (NAV). 

%De neste tre paragrafene var egentlig 2.1 - NAV. Flyttet hit
NAV is currently funding a Research-and-Development project in partnership with the IMTEL lab at NTNU investigating different extended reality (XR) technologies and attempting to determine their viability in helping those receiving support from NAV to find employment. As part of this, NAV and IMTEL has collaborated to create several virtual workplace experiences, with the goal to help young job seekers gain insight into different professions using immersive and interactive VR experiences \cite{IMTELinternships}. These VR experiences are called \textit{Virtual Internships}. 

This master thesis attempts to research the effects that multi-user experiences have on the learning efficacy of users using virtual internships to help them find employment.

The thesis and the underlying project are a part of the Virtual Internship project, and can be considered a branch project. From the Virtual Internship project we received access to VR expertise and existing codebases from previous projects, a VR lab with other students facilitating knowledge sharing, multiple VR seminars, and integration into a well established partnership between IMTEL and NAV.    
As part of the partnership the work was carried out in parallel with both IMTEL and directly with NAV. What this meant was that the development was done at the VR lab with guidance from a professor and researchers at IMTEL, while testing and obtaining requirement specifications was done in collaboration with NAV.    


%This master thesis attempts to research the effects that multi-user experiences have on the learning efficacy of the users attempting to find employment through NAV. 

Our contribution to the partnership consists of two main components. First, the evaluation results obtained over the course of the thesis. Perhaps the most important part, the results obtained will be used to decide whether or not NAV will pursue multi-user experiences for future Virtual Internship projects. Second is the artefact itself. The more general pieces of code and game objects can be used in other projects, assuming that future projects will contain multi-user experiences. This can include lobby systems, underlying network architecture, local features and other pieces of useful code. This means that parts of the project will be devoted to creating general scripts and \textit{prefabs} (see section \ref{sec:technologies} for more on prefabs) that can be used as building blocks for later development. As a Research-and-Development project, both of these components can be considered important contributions, and need to have high standards.


The research objective to explore a collaborative multi-user mode was defined by the results of the first phase of the Virtual Internship project \cite{NavVRrapport}. As such, IMTEL suggested a research direction to explore how multi-user experiences would affect Virtual Internship projects as end-users evaluated multi-user functionality as a one of the most wanted features in the report summarising the evaluation results \cite{NavVRrapport}.    



%Nyttig link :) https://www.nrk.no/vestfold/nytt-pilotprosjekt_-gaming-hjelper-johan-_22_-a-velge-riktig-yrke-1.14748067

The project's duration spans from September 2019 to June 2020. During the course of the project, we developed, researched, and tested a multi-user experience where young job seekers can try their hand at a virtual workplace, either together with a councillor from NAV or with one of their peers.

\section{Motivation}
%%Overview
VR has been around for decades and the term was first introduced introduced in the mid 1980's \cite{historyVR} with research centres such as NASA utilising head-mounted displays to create a virtual environment workstation. This is achieved by generating digital 3D content simulating different scenarios creating a greater feeling of presence than conventional displays. Recent technology advancements has provided the foundation for for low-cost and high quality devices available to the public. Major contributors such as Oculus with the Quest, HTC with the Vive and Valve with the Index has pushed the technology of VR headsets facilitating a convincing user experience.  

According to \textit{Greenlight Insights} \cite{forcastVR} the market size of VR and AR is predicted to grow by 7.7x from \$27 billion in 2018 to \$209 billion by 2022. This growth facilitates processes of research and development, pushing the boundaries of how we use and take advantage of such technologies.

Gaming has long been the major sector of VR applications but other sectors have seen investment such as education or healthcare. Applications such as the MIST system \cite{mccloy2001virtual} allows for training and assessment of surgical skills and other VR technologies has proven to have a positive effect on students' understanding of scientific concepts such as biology according to Shim et al. \cite{shim2003application}. As Kaufmann \cite{kaufmann2003collaborative} points out, several authors have suggested that the use of VR can raise interest and motivation in students with a high potential to enhance the learning experience.
Also, gamification of workplace training tasks or other learning tasks within medicine, safety training or history has proven to be successful at utilising VR and AR, made evident by the NAV FisheryVR application \cite{fishfarmNAV}. 

Using these technologies for workplace training and career opportunities has been increasingly popular. Major companies including  Walmart and Deutche Bahn is using VR as a mean to educate and present themselves as an innovative and advanced employer increasing the growth of job applications and training its associates for different operations with success \cite{vasilenko2019virtual}.    

As mentioned earlier, one of the most requested features during evaluation of phase 1 of the Virtual Internship project was multiplayer functionality. This is also evident in the 2019 paper by Henrichsen, both from the case study of existing applications as well as the research done for his own workplace application \cite{henrichsen2019engaging}.

Multiplayer interactive games have been shown to have multiple benefits and great educational potential \cite{ducheneaut2006alone} \cite{nardi2006strangers} \cite{steiner2006play}. There are strong indicators that multi-user functionality would be very useful for Virtual Internships. It could potentially solve the problem of a unengaging single user experience, and also lay the foundations for an interesting collaboration aspect between job seekers and career counsellors, a relatively unexplored research topic.   



\section{Problem Description}
The IMTEL lab has for last 3 years (since 2017) through different projects including other master thesis and bachelors assignments developed several applications of workplace training such as windmill electrician \cite{henrichsen2019engaging} or fishery- and construction worker. The aim of such applications has primarily been to create virtual environments simulating a real world workplace in order for job seekers to experience and gain valuable information from them.

These single-users virtual internships do not facilitate interaction between the job seeker and the career councillor. 
As such, this thesis will explore and evaluate collaborative mechanisms for Virtual Internships as employed by NAV, and how they can contribute to career guidance and potentially making them more engaging as multiplayer applications.


%How does the addition of collaboration to workplace experience applications affect the presence of the participants? What steps can be taken to improve the participants' feeling of presence?
\clearpage

\section{Research Questions}
The research questions for this thesis focus on how the addition of collaboration opportunities to a virtual internship application affects it in terms of in-person and remote career guidance, necessary features and rising challenges. They are as follows:
\label{RQ}
\begin{description}
    \item [Primary RQ:]\hfill \\
    “How does collaboration in virtual reality workplaces contribute to the career guidance of young job seekers?” 
    \item [Secondary RQ1:]\hfill \\
    %“What are the effects of collaborating in virtual reality  with a mentor compared to a peer?”
    "How is self-efficacy affected by seeker-seeker collaboration compared to seeker-councillor collaboration in virtual reality?"
    \item [Secondary RQ2:]\hfill \\
    %"What is essential for collaborative virtual reality to be applied effectively for remote career guidance?"
    "Which features are effective at facilitating collaborative virtual reality for remote career guidance?"
    \item [Secondary RQ3:]\hfill \\
    "What type of collaborative features are technologically feasible for virtual reality workplaces?" *
    \item  [Secondary RQ4:]\hfill \\
    %"What are the challenges of implementing collaborative features in an ongoing virtual reality project?"
    "What challenges arise when implementing collaborative features in an ongoing single-user virtual reality project?"
    
\end{description}

\vspace{2cm}
*\textit{By technologically feasible we mean features (such as laserpointer or hand gesticulation) that can be developed and carried out to fulfil its objective. }

%\section{Overarching goals}
%TODO: Write ingress
%\label{overarchingGoals}
%\begin{description}
%    \item [Technical prerequisites] \hfill \\
%    Determine which technical prerequisites are needed when implementing collaboration in VR in relation to the ongoing NAV project.
%    \item [Reusable assets] \hfill \\
%    Determine a set of required and reusable assets for multi-user experiences.
%\end{description}

\cleardoublepage
