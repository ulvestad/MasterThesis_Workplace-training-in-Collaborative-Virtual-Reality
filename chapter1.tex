%===================================== CHAP 1 =================================

\chapter{Introduction}

This project is about the use of virtual reality for creating a collaborative workplace training experience. In this chapter we will present the context, motivation and research questions of the project. 

\section{Context}
\label{section:context}
This project was part of a master's program in Informatics at the Norwegian University of Science and Technology (NTNU) in Trondheim. It was a collaboration between the Department of Computer Science (IDI), the Department of Education and Lifelong Learning (Innovative Immersive Technologies for Learning, IMTEL) and the Norwegian Labour and Welfare Administration (NAV). 

%De neste tre paragrafene var egentlig 2.1 - NAV. Flyttet hit
NAV is currently funding a pilot project in partnership with the IMTEL lab at NTNU investigating different XR technologies and attempting to determine their viability in helping those receiving support from NAV to find employment. As part of this, NAV and IMTEL has collaborated to create several workplace experiences to introduce new working environments to potential employees in a safe and educational environment.

This thesis and underlying project is a part of this, and can be considered a branch project. While normal development continues on virtual workplaces, this project attempts to research the effects that multi-user experiences have on the learning efficacy of the users attempting to find employment through NAV's programs. 

As a part of this, a request has been made that if possible, all artefacts should be made to be as general as possible, with the possibility to use them for the development of later projects, should that be desired. What this means is that some time of the project will be devoted to crafting general scripts and \textit{prefabs} (See section \ref{sec:technologies} for more on prefabs.) that can be used as building blocks for later development.

%Nyttig link :) https://www.nrk.no/vestfold/nytt-pilotprosjekt_-gaming-hjelper-johan-_22_-a-velge-riktig-yrke-1.14748067

The project's duration was September 2019 - May 2020. During the course of the project, we have developed, researched, and tested a multi-user experience where young job seekers can try their hand at a virtual workplace, either together with a councillor from NAV or with one their peers. The findings will be presented in this paper.

\section{Motivation}
%%Overview
Virtual reality (VR) has been around for decades and the term was first introduced introduced in the mid 1980's \cite{historyVR} with research centres such as NASA utilising head-mounted displays to create a virtual environment workstation. Recent technology advancements has provided the foundation for for low-cost and high quality devices available to the public. Major contributors such as Oculus with the Quest, HTC with the Vive and Valve with the Index has pushed the technology of VR headsets facilitating a convincing user experience.  

According to \textit{Greenlight Insights} \cite{forcastVR} the market size of VR and AR is predicted to grow by 7.7x from \$27 billion in 2018 to \$209 billion by 2022. This growth facilitates processes of research and development, pushing the boundaries of how we use and take advantage of such technologies.

Gaming has long been the major sector of VR applications but other sectors has seen have seen investment such as education or healthcare. Applications such as the MIST system \cite{mccloy2001virtual} allows for training and assessment of surgical skills and other VR technologies has proven to have a positive effect on students understanding of scientific concepts such as biology according to Shim et al. \cite{shim2003application}. As Kaufmann \cite{kaufmann2003collaborative} points out, several authors have suggested that the use virtual reality can raise interest and motivation in students with a high potential to enhance the learning experience.
Also, gamification of workplace training tasks or other learning tasks within medicine, safety training or history has proven to be successful at utilising VR and AR such as evident by the NAV fish-farm application \cite{fishfarmNAV}.


%origin


%Technology / immersion


%History/education


%NAV & workplace training


%Notes: 


%

\section{Problem Description}
The IMTEL lab has for the couple of years through different projects including other master thesis and bachelors assignments developed several applications of workplace training such as windmill electrician \cite{henrichsen2019engaging}, salmon fish-farm \cite{} \textcolor{red}{MANGLER} and construction work \cite{}  \textcolor{red}{MANGLER}. The aim of such applications has primarily been to create virtual environments simulating a real world workplace so that job seekers can experience and gain valuable information from them.
However, there has been comments from participants and NAV employees that these VR applications can make the player feel isolated in the "VR world" and that there has been some difficulty in mentoring the player as an observer. With that in mind this project aims to implement collaboration in VR in hopes of improving the feeling of presence for the user, as well as improving and accelerating the learning effect. 

%How does the addition of collaboration to workplace experience applications affect the presence of the participants? What steps can be taken to improve the participants' feeling of presence?

\section{Research Questions}
\label{RQ}
\begin{description}
    \item [Primary RQ:]\hfill \\
    “How can collaboration in virtual reality (VR) contribute to the workplace experience of young job seekers?” 
    \item [Secondary RQ1:]\hfill \\
    “What are the effects of collaborating in virtual reality (VR) with a mentor compared to a peer?”
    \item [Secondary RQ2:]\hfill \\
    "What type of tasks are suitable for learning in a collaborative virtual environment?"
\end{description}


\cleardoublepage
