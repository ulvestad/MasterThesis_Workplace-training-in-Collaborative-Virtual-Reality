%===================================== CHAP 4 =================================

\chapter{Research Design and Methodology}
Over the course of this project, both quantitative and qualitative data gathering methods and analysis were employed. Both of the methods offered something valuable to the project, and using both allowed us to gather the data best suited for the thesis. In the early stages quantitative data, such as surveys, were used to gather opinions and thoughts from potential users of the program. Using these surveys, we could create a base to build from, and develop a better prototype. With this prototype finished, more qualitative measures were used to gather more specific and more detailed data.

\section{Design and Creation Research}
\label{sec:designCreationResearch}
One of the more common research methodologies in computer science, Design and Creation Research uses common development techniques and adapts them for research. It focuses on development of new IT products, also called artefacts\cite{oates2005researching}. In the case of this paper, the artefact being developed is a prototype aiming to figure out whether or not collaboration is conducive to workplace training in virtual reality. This type of research can be split into a few different types, mostly depending on whether the artefact itself is the end goal, a vehicle for research or if the focus is on the development process itself. The most relevant case for this project is the second case. To expand upon this one slightly, to say that the artefact is a vehicle for the research means that while the artefact itself is important, its \textit{usage} in real life is what's really important. By developing something that can be used in real life, it is possible to point to something concrete and measure the perceived effects the artefact has on the system it is introduced to.

While the artefact in this project is a vehicle for the research, the aim is to have one of the pre-existing workplace training applications working with collaborative elements at close to 100\% functionality, so that others may continue developing the application later with relative ease. The most important part is still to see whether or not the end product can heighten the efficacy of the VR workplace training project.


\section{Development methodology}
The primary type of development used in the project can best be described as a variant of agile software development. Agile development refers to certain principles that were written down in the 2001 article \textit{Manifesto for Agile Software Development}\cite{beck2001manifesto}. This manifesto decries the old method of rigid development that was very resistant to change. Over the course of this project, feedback is sought at every level, hoping to improve the software both for the customer and the users, and that means that it must be open to change when it needs to.

There will be multiple iterations and phases of the project, each building upon the last. Three major phases were roughly planned out at the beginning of the project, with each containing multiple iterations as priorities shift based on feedback. These three phases are discussed in more detail in chapters \ref{chap:phase1}, \ref{chap:phase2} and \ref{chap:phase3}, respectively.



\section{User testing}
\subsection{How the tests were performed}
User tests are an important part of agile development. They are one of the primary means of gathering relevant feedback from the people who will actually be using your product. A lot of care goes into creating the tests and making sure that the data extracted from them is sound and valid. The tests needs to specific enough to showcase or test the aspect you are focusing on, but not so specific that it becomes an unnatural situation that does not represent the actual product.


\subsection{The Surveys}
In the beginning phases of the project, it was deemed important to get an overview of the general attitude of the target audience when it came to VR. 


\subsection{Expert tests}
Since most of the work of this project was done at the IMTEL lab, numerous opportunities presented themselves for us to gather feedback and advice. Throughout the year, people from many different fields and disciplines stopped by the lab for various reasons, and most were quite open to some discussion about the project. The multidisciplinary nature of the project meant that even if they were not necessarily well versed in every aspect, they still had some valuable input for us to consider.

As such, during the course of the project, we often got to test the current iteration on people from different people. These tests may have been more informal than the standard user tests that were performed, they offered a lot of insight that we may not have considered on our own.




\cleardoublepage