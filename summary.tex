\begin{center}
\vspace*{1mm}
\section*{\Huge Abstract}
\addcontentsline{toc}{chapter}{Abstract}	
\vspace*{0.7cm}
\end{center}

%USING => OMRC. Objectives, Methodology, Results, and Conclusion.


%Background
\noindent The use of virtual reality technologies for workplace training and education is increasingly popular. Using virtual reality with the goal of helping young job seekers to gain insight into different professions is an ongoing project called  \textit{Virtual Internship} developed in partnership by the Norwegian University of Science and Technology and the Norwegian Labour and Welfare Administration.

%Objective \ %Method
This thesis will look into the effects that collaboration has on virtual reality career guidance, both remote and co-located. By using an already existing \textit{Virtual Internship} application and modifying it to allow for collaborative work, we will investigate whether or not collaborative features are conducive to career guidance in virtual reality. For software development we adapted an agile development method with three iterative phases. Both qualitative and quantitative data  was utilised to help answer the research questions.   

%Results \ %Conclusion
Collaborative virtual reality was found to lead to increased user engagement and self-efficacy. It also eased the process of providing guidance during the use of a \textit{Virtual Internship} system. For remote career guidance we found features that simplify communication to be imperative.
Adding multi-user functionality in an existing single-user virtual reality application can vary in difficulty, but the developed framework in this thesis should help ease future development so that existing applications can accommodate collaborative mechanisms. 



\bigskip

\noindent \emph{\textbf{Keywords:} Virtual Reality, Collaboration, Career Guidance, Remote Career Guidance, Multi-user Virtual Reality}

\cleardoublepage