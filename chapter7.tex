%===================================== CHAP 7 =================================

\chapter{Phase 3: Last Requirements and Finishing Up} \label{chap:7}
\label{chap:phase3}

\section{Planning and Changes}
Phase 3 was originally meant to contain changes and finishing touches based on the feedback received during phase 2. However, just as the main testing was about to begin, Norway took measures to combat Covid-19 (see Section \ref{section:covid19}), effectively preventing us from going through with the plans that were laid out for this phase. Coupled with an enforced work from home situation, and the plan needed to be changed drastically. Primarily, some triage had to be done as to figure out which parts of the application were the most necessary for remote data gathering. Since the application is a proof of concept, it was never that important that it ran without supervision, or that multiple server instances could be ran at the same time, since we were always present to run the tests and could manage all of this manually. Now that could no longer be assured. This lead to some significant changes in the planned schedule, and more features needed to be developed.

\subsection{Changes}
\begin{description}
    \item [Change 1]\hfill \\
    Support for multiple concurrent rooms must be made. 
    \item [Change 2]\hfill \\
    Artefact must be changed to a file runnable by end-users with minimal to no input from developers.
    \item [Change 3]\hfill \\
    Features need to be complete and usable, or potentially removed if their unfinished state interferes with normal use of artefact.
\end{description}

\subsection{Final Requirements}
\begin{enumerate}
  \item [\textbf{F1}] Users should be able to interact with the environment.
\end{enumerate}



\subsection{Development Decisions}
\subsubsection{VR}

\section{Implementation}

\subsubsection{Lobby Supporting Different Applications}
As part of the larger collaboration project between IMTEL and NAV, it's natural for new projects to build on the older ones, like this one did. To keep this going, a lot of effort went into making sure the artefact is reusable and easy to build further should the need arise to do so. As part of this, the starting screen of the application has received a new feature, where the user can select which application they would like to start. To demonstrate the possibility, a demo room was made, so that the user can correctly select different rooms and see the feature in action. It also servers to demonstrate how to implement a new scene for future developers who may use the lobby system.

Skriv om hvordan lobby støtter flere apper. Vi har laget en demo app i tilegg til autoVR. Dette er fordi det svarer til siste RQ om : Secondary RQ4:”What  are  the  challenges  of  implementing  collaborative  features  in  an  ongoing project?"



\subsection{Improvements}
\subsubsection{Usability}
\subsubsection{Other}

\subsection{Unimplemented Features}
\subsubsection{Feature 1}
\subsubsection{Other}


\section{Third Evaluation}
More about a Kompentanse Norge og presentasjon over Zoom + Microsoft form etc. 

\subsection{Expert test data}
%Graph data: 

%Learning

%Other improvements

%VR mode vs. desktop

%Expert feedback


\subsection{Analysis}

\subsubsection{Learning outcome}

\subsubsection{VR mode vs. desktop}

\subsubsection{Expert evaluation}


\cleardoublepage