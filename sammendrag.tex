\begin{center}
\vspace*{1mm}
\section*{\Huge Sammendrag}
\addcontentsline{toc}{chapter}{Sammendrag}	
\vspace*{0.7cm}
\end{center}

Bruken av virtuell virkelighets teknologier for opplæring på arbeidsplass og utdanning blir stadig mer populært. Å bruke virtuell virkelighet med hensikt til å hjelpe unge jobbsøkere å få innsikt i forskjellige yrker er et pågående prosjekt kalt \textit{Virtual Internship} utviklet sammen med Norges Teknisk-Naturvitenskapelige Universitet og NAV.

Denne oppgaven undersøker effektene samarbeid har på karriereveiledning i virtuell virkelighet, både over avstand og samlokalisert. Ved å modifisere en allerede eksisterende \textit{Virtual Internship} applikasjon for å legge til samarbeidsfunksjonalitet vil vi se hvorvidt samarbeidsmekanismer hjelper eller hindrer karriereveiledning i virtuell virkelighet. For programvareutvikling tilpasset vi en smidig utviklingsmetode med tre iterative faser. Både kvalitative og kvantitative data ble brukt til å svare på forskningsspørsmålene. 

Samarbeid i virtuell virkelighet ble funnet å føre til økt brukermedvirkning og mestringstro. Det lettet også prosessen med å gi veiledning under bruk av \textit{ Virtual Internship } systemet. For ekstern karriereveiledning fant vi funksjoner som forenkler kommunikasjon for å være helt nødvendig. Å legge til flerbrukerfunksjonalitet i et eksisterende virtuell virkelighet program designet for én bruker kan variere i vanskelighetsgrad, men rammeverket utviklet i denne oppgaven kan hjelpe til med å lette fremtidig utvikling slik at eksisterende applikasjoner kan bedre legge opp til samarbeid.


\bigskip

\noindent \emph{\textbf{Nøkkelord:} Virtuell Virkelighet, Samarbeid, Karriereveiledning, Avstandsbasert Karriereveiledning, Flerbruker Virtuell Virkelighet}

\cleardoublepage