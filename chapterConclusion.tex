%===================================== CHAP 8 =================================

\chapter{Conclusion}
In this chapter we present our contributions, conclusions and recommendations for future work.

\section{Contributions}

Through this project, we have created a series of artefacts that can be used for future work at the IMTEL lab.  They will serve as additions to the research so that decisions about multi-user and collaboration aspects can be made for future projects.   

\begin{itemize}
    \item Extensible lobby system
    \item PUN2 network prefabs
    \item Prototype collaborative VR career guidance system
    \item Strong support for collaboration in career guidance
\end{itemize}

%Artefact
%- Lobby system
%- Prefabs for VR/Desktop users
%- can be used as a template for future work at IMTEL and NAV

%Research-and-Development to the IMTEL and NAV partnership:
%- findings in this thesis contributes to this

\section{Conclusion}
This thesis has looked at several aspect including the effects of collaboration and remote career guidance, but also more technical aspects related to development and challenges. 
There is a clear benefit and value of having collaboration possibilities in virtual environments for career guidance. It yields positive effects in regards to engagement of the users and decision learning. Collaboration also impacts young job seekers self-efficacy and social skills. 

For remote career guidance we found the most important features to be the ones that help simplify communication such as deictic gestures, laser pointers and voice communication. Avatars must be behaviourally realistic, not necessarily visually. The application must also be easy to install and intuitive to use to make sure that users do not quit before getting started. These features are also technologically feasible for VR workplaces.

Single-user VR application are to begin with not developed for collaboration. This means that you can make a number of assumptions and rules that no longer work in a collaborative environment. Depending on how much code is based around there only ever being one player is made, significant challenges may arise. If the code is not documented, a lot of time may need to be spent understanding the old code. However, it is quite possible to add multi-user functionality with varying degree of difficulty. Complexity, clean code and the number of elements which needs networking all impact this. As such a framework such as the one we have developed can help accommodating collaborative features. In essence, the most important part is to adhere to principles such as high cohesion - low coupling and other similar coding practices. Avoid shortcuts where you can and make sure artefacts are well documented.





%\textit{Conclusion}

%Hoved RQ
    %JA, collab er positivt for karrieveildening
        % engagement
        % decision learning

% RQ1
    % kan ikke besvares, altså ingen konklusjon
    
% RQ2
    % flere features som er viktig
        %voice
        %deictic guesture
        %avatar
    
    %konspeter
        % presence (også user representaiton as avatar)
        % ease of use (SUS score low)
        % CSCL/workspace awereness
        
% RQ3
    % feasable
        %voice
        %deictic guesture
        %avatar

% RQ4
    % clean code
    % 

    

\section{Future Work}
Future work and suggestion for future projects is given here. These suggestion are based on the authors experience from the research project presented in this paper. 

\label{section:futureWork}

\subsection{Secondary RQ1}
While our research found collaboration in VR impacts career guidance, this thesis can not provide valid discussion or conclusions on whether or not this changes depending on the use of seeker-seeker collaboration or seeker-counsellor collaboration. We therefore suggests that future work could investigate the self-efficacy gained from of seeker-seeker collaboration compared to seeker-counsellor. The artefacts from this thesis could be used as platform for testing.  

\subsection{New Hardware}
In order to make it easier to test quickly and efficiently, we recommend making use of hardware such as the Oculus Quest \cite{hillmann2019comparing} in order to minimise the amount of hardware needed to run the applications. Such hardware does not rely an a separate computer to contain the VR engine, meaning they are standalone VR devices (see section \ref{section:VRhardware}). Our experience while testing showed that many schools and corporations had bought Oculus Quests in large numbers, but had relatively few standard VR headsets, much less computers to run them on. 

Alternatively, more can be done to make use of the more expensive "standard" VR headsets, such as the Valve Index, with increased feedback and detailed hand gestures to increase realism and user presence.

\subsection{Desktop Mode}
While the desktop mode was not originally planned, its inclusion garnered positive feedback and a desire to see more functionality added to it. Expanding its features and allowing greater interaction with the scene while in this mode could increase the availability and usability of the artefact, and allowing a larger variation of use scenarios. The time allotted for this project did not allow for a thorough examination of potential features, but preliminary feedback would indicate that the most desired features are a laser pointer, the ability to interact with objects and more highlighting opportunities.

Depending on the extension of this mode, it may also be relevant to change the avatar and movement mode to something more akin to the VR players, i.e., a humanoid avatar with movement affected by physics constricted to ground level. It would be interesting to see how this affected collaboration and whether or not it is jarring for a VR user to collaborate with a humanoid avatar with less natural movement and behaviour like a desktop avatar would bring to the table.

\cleardoublepage